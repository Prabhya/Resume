\documentclass[10pt,a4paper]{extarticle}
%\moderncvstyle{banking}                            % style options are 'casual' (default), 'classic', 'oldstyle' and 'banking'
%\moderncvcolor{blue} 
\usepackage[letterpaper, total={7.4in, 10.2in}]{geometry}
\usepackage{amsmath}
\usepackage{soul}
\usepackage{enumitem}
\usepackage{titlesec}
\titlespacing*{\section}{0pt}{0.3\baselineskip}{0.2\baselineskip}
\usepackage{xcolor}
\usepackage[colorlinks = true,
            linkcolor = blue,
            urlcolor  = blue,
          %  citecolor = blue,
            ]{hyperref}
  %\usetheme{Copenhagen}
 % \usecolortheme{orchid}%beaver
%\hypersetup{   
  %  colorlinks =true,
 %   urlcolor=blue,
%}
\newcommand{\changeurlcolor}[1]{\hypersetup{urlcolor=#1}} 
%\usecolortheme{orchid}
\urlstyle{same}
\usepackage{amssymb}
%\sectionfont{11pt}
\title{Resume}
%\maketitle
\begin{document}

\begin{center}
\textbf{\huge {Prabhjeet Singh Arora}}\\
\small{Control Systems $|$ Optimization $|$ Path Planning $|$ Unmanned Vehicle Systems $|$ Robotics}\\
\small{Arlington, TX-76013 $|$ +1 (682)227-5419 $|$ \href{prabhjeet.arora@mavs.uta.edu}{\ul{prabhjeet.arora@mavs.uta.edu}} $|$ \changeurlcolor{cyan}\href{http://www.linkedin.com/in/prabhjeet-arora}{\ul{linkedin.com/in/prabhjeet-arora}} $|$ \changeurlcolor{blue}\href{https://github.com/Prabhya}{\ul{github.com/Prabhya}}}
\end{center}

\section*{\colorbox{gray!10}{\makebox(250,9){\textcolor{blue!65}{Education\hfill}}}}
\textbf{Master of Science | Mechanical Engineering} | \textbf{University of Texas at Arlington, TX}\hfill\textbf{Aug'16 - Dec'18}
\vspace{-0.5em}
\begin{itemize}[leftmargin = 0.6cm]
\setlength\itemsep{-0.2em}
\item Thesis: Reachable Set Computation and Analysis for Perturbed Linear Systems \hfill(GPA - 3.78/4) 
%\item Course: Classical Methods of Control Systems Analysis and Synthesis, Optimal Estimation of Dynamic Systems, Engineering Analysis I and II, Fluid Dynamics, Finite Element methods, Structural Aspect of Design, Combustion, Micro/Nano fabrication.
\end{itemize}
\textbf{Bachelor of Technology | Mechanical Engineering} | \textbf{Indian School of Mines, India} \hfill\textbf{Aug'11 - May'15}
\vspace{-0.5em}
\begin{itemize}[leftmargin = 0.6cm]
\setlength\itemsep{-0.2em}
\item Final year project: Bachelor's Final year project - Study of Regenerative braking system
%\item Control systems related Course: Optimization, Linear Programming, Fluid power and control systems.
\end{itemize}
%\end{itemize}


\vspace{-0.2cm}

\section*{\colorbox{gray!10}{\makebox(250,9){\textcolor{blue!65}{Key Skills\hfill}}}}
\begin{description}[align=left,labelwidth=1.8cm,leftmargin = 2.25cm]
\setlength\itemsep{-0.25em}
\item[Language ] - C++, Python
\item[Softwares ] - ROS, Visual Studio code, MATLAB, Simulink, Solidworks, AutoCAD, Abaqus, Excel, LaTeX, MS Office
\item[Skills ]  - Linear and Nonlinear Control, Optimal Control, Optimal Estimation, Optimization, Reachable Sets, Functional Analysis, Path Planning, Motion Planning
\end{description}
\vspace{-0.3cm}

\section*{\colorbox{gray!10}{\makebox(250,9){\textcolor{blue!65}{Experience\hfill}}}}
\textbf{Research Assistant} | \textbf{LEARN Laboratory (UTA), Arlington, TX} \hfill \textbf{Jan 2019 - Present}
\vspace{-0.5em}
\begin{itemize}[leftmargin = 0.6cm]
\setlength\itemsep{-0.2em}
\item Developed potential field based path planner for 7 degree of freedom Baxter robot arm and executed real time planning
\item Decentralized the planning method for last 3 degree of freedom
\item Reduced the computation time required for motion planning
\end{itemize}
\textbf{Research Assistant} | \textbf{Aerospace Laboratory (UTA), Arlington, TX} \hfill \textbf{Jul 2017 - Dec 2018}
\vspace{-0.5em}
\begin{itemize}[leftmargin = 0.6cm]
\setlength\itemsep{-0.2em}
\item Theorized the conditions for discretizing nonlinear system using Euler 1-step discretization %for computation purpose
\item Created procedure to handle discretized nonlinear systems with discrete input and produced numerical methods
\item Developed algorithms to compute reachable sets and invariant sets of discrete nonlinear systems %, and determined the 
\item Further improved the algorithms to compute overapproximated reachable sets of complex nonlinear systems
\item Reduced space usage by 50\% and reduced computation time significantly
\end{itemize}
\textbf{Graduate Teaching Assistant} | \textbf{University of Texas at Arlington, Arlington, TX} \hfill \textbf{Jan 2018 - Jun 2018}
\vspace{-0.5em}
\begin{itemize}[leftmargin = 0.6cm]
\setlength\itemsep{-0.2em}
\item Tutored students undertaking the course - Classical Methods of Control Systems Analysis and Synthesis
\item Proctored exams, as well as, graded exam and homework
\end{itemize}
\iffalse
\textbf{Junior Engineer} \hspace{4.95cm} \textbf{A. R. Engineers and Consultants} \hfill\textbf{Aug 2015 - Jan 2016}
\vspace{-0.5em}
\begin{itemize}[leftmargin = 0.6cm]
\setlength\itemsep{-0.2em}
\item My work entailed design interpretation before manufacturing the parts in CNCs, with particular attention to quality control (ISO standards)
\item After the parts were manufactured, final check to affirm no errors while manufacturing is carried out
\end{itemize}
\fi
\textbf{Hydraulics Engineering Intern} | \textbf{EATON Corporation, Pune, India} \hfill\textbf{May 2014 - Jul 2014}
\vspace{-0.5em}
\begin{itemize}[leftmargin = 0.6cm]
\setlength\itemsep{-0.2em}
\item Worked on dynamic system modelling and implemented hydraulic power and controls systems in stacker-reclaimer project to design optimized systems
%Worked on dynamic system modelling for Stacker-Reclaimer project.
%\item Primary focus was on implementing hydraulic power and control systems in stacking, reclaiming and luffing mechanism.
%\item Implementation was accomplished to fulfill customer requirements, while utilising standard actuators and designing optimized system.
\end{itemize}

\vspace{-0.2cm}


\section*{\colorbox{gray!10}{\makebox(250,9){\textcolor{blue!65}{Publication\hfill}}}}
\textbf{Reachable Set Computation and Analysis for Perturbed Linear Systems} : \href{http://hdl.handle.net/10106/27661}{\ul{http://hdl.handle.net/10106/27661}}
\vspace{-0.6em}
\begin{itemize}[leftmargin = 0.6cm]
\setlength\itemsep{-0.2em}
%\item Sufficient Condition of reachability - for perturbed linear discrete systems (Nonlinear Systems) was evaluated. Allowing for segregating systems whose reachable sets exist.
%\item Algorithm - was constructed to compute reachable sets for perturbed linear discrete systems. The algorithms were further built into programs in MATLAB for verification of.
%\item Analysis and computation - of reachable sets was done for various perturbed systems.
\item Computed reachable set and invariant set of Li\'enard systems of Lipschitz kind with bounded control input (Matlab)
\vspace{-0.6em}
\begin{itemize}[leftmargin = 0.3cm]
\setlength\itemsep{-0.2em}
\item Provided information of all the states the system can achieve
\end{itemize}
%\item Computed and analyzed reachable set and invariant set of Li\'enard systems of Lipschitz kind with bounded control input - the reachable set was computed in Matlab, to provide information of all the states the system can achieve, given that the input is limited
\item Computed overapproximated reachable set of errors in multi rotor system (Matlab)
\vspace{-0.6em}
\begin{itemize}[leftmargin = 0.3cm]
\setlength\itemsep{-0.2em}
\item Provided the ``region of collision" due to nonlinear disturbance of first-order aerodynamic effect
\end{itemize}
%\item Computed and analyzed reachable set of errors of multi rotor system under first-order aerodynamic effects (simulation in Matlab) - the reachable set of errors provided the ``region of risk" where collision could occur due to nonlinear disturbances
%(with and without compensation) was , the target was to achieve an error set around the centre of mass of Multi rotor system for safety.
\end{itemize}
\vspace{-0.2cm}



\section*{\colorbox{gray!10}{\makebox(250,9){\textcolor{blue!65}{Projects\hfill}}}}
%\ul{\textbf{Academic Projects}}\\
\textbf{Projects on Optimal Estimation of Dynamic Systems}
\vspace{-0.4em}
\begin{itemize}[leftmargin = 0.6cm]
\setlength\itemsep{-0.2em}
\item Tracked a maneuvering target with sensor data (Matlab, Python)
\vspace{-0.6em}
\begin{itemize}[leftmargin = 0.3cm]
\setlength\itemsep{-0.2em}
\item Implemented Extended Kalman Filter, Unscented Filter and Particle Filter in Matlab and Python
\item Estimated position of a maneuvering target with error within Three-Sigma Limits
\item Estimation time and data required were compared
\end{itemize}
%\item Implemented Extended Kalman Filter, Unscented Filter and Particle Filter using Matlab and estimated the position of a maneuvering target, with error within Three-Sigma Limits. The methods were then compared, based on time and data required for estimation
%\item \ul{Track a maneuvering target using Extended Kalman Filter, Unscented Filter and Particle Filter}: All three methods were implemented using Matlab, to estimate the position of a maneuvering target, with error within Three-Sigma Limits. The methods were then compared, based on time and data required for estimation.
\item Estimated unknown parameter of Van der Pol system (Matlab, Python)
\vspace{-0.6em}
\begin{itemize}[leftmargin = 0.3cm]
\setlength\itemsep{-0.2em}
\item Implemented MMAE approach, IMM estimator in Matlab and Python
\item Estimated unknown parameter with error within Three-Sigma Limits
\end{itemize}
%\item Estimated unknown parameter using MMAE approach and IMM estimator by implementing them in Matlab and successfully estimated the damping coefficient of Van der Pol system, with the error within Three-Sigma Limits for respective methods
%\item \ul{Estimate unknown parameter using MMAE approach and IMM estimator}:  Both the methods were implemented using Matlab, to successfully estimate the damping coefficient of Van der Pol system, with the error within Three-Sigma Limits for respective methods.
\item Achieved sensor data fusion (Matlab)
\vspace{-0.6em}
\begin{itemize}[leftmargin = 0.3cm]
\setlength\itemsep{-0.2em}
\item Implemented Covariance Intersection in Matlab
\item Performed sensor data fusion for given data from multiple sensor models with different error covariances
\item Produced optimal error covariance
\end{itemize}
%\item Implemented Covariance Intersection in Matlab for given data from multiple sensor models with different error covariances and achieved sensor data fusion and produced most optimised error covariance
%\item \ul{Achieve sensor data fusion by implementing Covariance Intersection (CI) method}: CI method was implemented using Matlab for given data from multiple sensor models with different error covariances, to produce most optimised error covariance.
%- Synthetic measurements were created for Van der Pol system with noise.
%- The model was presented keeping damping coefficient as unknown parameter.
%- MMAE approach and IMM estimator were implemented in MATLAB to estimate the parameter.
%- Both the methods successfully estimated the damping coefficient, with the error within Three-Sigma Limits for respective methods.
\end{itemize}
\textbf{Project on Classical Control Theory}
\vspace{-0.5em}
\begin{itemize}[leftmargin = 0.6cm]
\item Designed a feedback controller to eliminate flight disturbance during auto pilot of helicopter using classical methods of control systems, reducing steady state error to 4\%
%\ul{Designed controller to eliminate flight disturbance during auto pilot of helicopter using classical methods of control systems}: designed a feedback controller to reduce steady state error to 4\%, using classical methods.
\end{itemize}
\vspace{-0.4em}
%\textbf{Design Related Projects}
%\begin{itemize}
%\item Analysis and Design Modification on Abaqus for retaining ring and Reed Valve at UT Arlington.
%\item Comparison of Traditional Fins with Nanofibre fins.
%\end{itemize}







\iffalse
\vfill

\section*{\colorbox{gray!10}{\makebox(250,9){\textcolor{blue!65}{Extra Activities\hfill}}}}
\vspace{0em}
\begin{itemize}
\setlength\itemsep{-0.2em}
\item Participated in workshop on Image Processing and Robotics.
\item Participated in workshop on Catia V5.
\item Participated in workshop on RC Aeroplane modelling and control.
\end{itemize}

\vspace{-0.45cm}
\fi

\end{document}